\hypertarget{research}{%
\section{Research}\label{research}}

\hypertarget{how-will-a-synchronization-server-be-build}{%
\subsection{How will a synchronization-server be
build}\label{how-will-a-synchronization-server-be-build}}

In this sub-chapter will be explained how a synchronization-server will
be build.

\hypertarget{how-will-the-synchronization-server-save-the-data-of-the-finite-state-machines}{%
\subsubsection{How will the synchronization-server save the data of the
finite-state
machines}\label{how-will-the-synchronization-server-save-the-data-of-the-finite-state-machines}}

A synchronization-server will hold the finite-state machine data in a
new container, this container will be called finite-state machine
derivative, or derivative in short.\\
This derivative will contain the necessary information the
synchronization-server needs to determine the next action that will be
executed by the system.\\
This information will be:

\begin{itemize}
\tightlist
\item
  Sensitivity-list, the sensitivity-list of the finite-state machine
\item
  Alphabet, the alphabet of the finite-state machine
\item
  Endpoint, the Minix endpoint used for sending messages
\item
  Index, this index of this derivative
\end{itemize}

This information is enough for the synchronization-server to determine
the next action.

These derivatives will be stored in an array, an array is used because
when using indexes it has a \(O (1)\) lookup time.

\hypertarget{how-will-the-synchronization-server-determine-the-next-action-for-the-finite-state-machines}{%
\subsubsection{How will the synchronization-server determine the next
action for the finite-state
machines}\label{how-will-the-synchronization-server-determine-the-next-action-for-the-finite-state-machines}}

For an action to be eligible for execution the following must be true:\\
When an action is present in a finite-state machine his alphabet its
must also be present in the finite-state machine his sensitivity-list.
This is true for every finite-state machine controlled by the
synchronization-server.

This means that when one or more finite-state machines have an action in
their alphabet but not their sensitivity-list this action cannot be
executed.

The following steps will be taken to determine which action can be
executed:

\begin{itemize}
\tightlist
\item
  A set will be filled with every action that is currently present in
  the sensitivity-lists.
\item
  A new set will be filled with every action that cannot be executed.
\item
  The second set, containing the actions that cannot be executed, will
  be subtracted from the first set, all currently present actions.
\item
  The result set will contain only the actions that can currently be
  executed.
\item
  A random action will be chosen from the remaining set, this will be
  the action that will be executed.
\end{itemize}

Now the action that the finite-state machines will execute has been
determined.

\hypertarget{how-will-the-synchronization-server-determine-the-finite-state-machines-that-will-execute-an-action}{%
\subsubsection{How will the synchronization-server determine the
finite-state machines that will execute an
action}\label{how-will-the-synchronization-server-determine-the-finite-state-machines-that-will-execute-an-action}}

In the previous sub-question the action that will be executed was
determined, in this sub-question the finite-state machines that will
execute this action will be determined:

\begin{itemize}
\tightlist
\item
  A new set is created, this set will hold the derivatives that will
  execute the selected action.
\item
  The array of derivatives will be iterated.
\item
  If the sensitivity-list and alphabet of the derivative contains the
  selected action the derivative can execute the action and is added to
  the designated set.
\item
  After iterating every element in the array of derivatives the result
  set contains all derivatives that can execute the action.
\end{itemize}

Now a set of derivatives that can execute an action have been
determined, and because the derivatives represent a finite-state
machine, it is also determined which finite-state machines can execute
the action.
