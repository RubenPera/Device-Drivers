\hypertarget{research}{%
\section{Research}\label{research}}

\hypertarget{what-can-the-functionality-of-the-device-drivers-be-implemented}{%
\subsection{What can the functionality of the device-drivers be
implemented}\label{what-can-the-functionality-of-the-device-drivers-be-implemented}}

\hypertarget{how-can-the-functionality-of-device-driver0-be-implemented}{%
\subsubsection{How can the functionality of device-driver0 be
implemented}\label{how-can-the-functionality-of-device-driver0-be-implemented}}

As specified in the Requirements the functionality of device-driver0
will be to continuously read from stdin.\\
As the application that will be build in this paper will be build with
C, for reading the stdin the basic commands are:

\begin{longtable}[]{@{}llll@{}}
\toprule
\begin{minipage}[b]{0.11\columnwidth}\raggedright
Name\strut
\end{minipage} & \begin{minipage}[b]{0.23\columnwidth}\raggedright
Explanation\strut
\end{minipage} & \begin{minipage}[b]{0.30\columnwidth}\raggedright
Input arguments\strut
\end{minipage} & \begin{minipage}[b]{0.25\columnwidth}\raggedright
Return value\strut
\end{minipage}\tabularnewline
\midrule
\endhead
\begin{minipage}[t]{0.11\columnwidth}\raggedright
getchar\strut
\end{minipage} & \begin{minipage}[t]{0.23\columnwidth}\raggedright
Reads the next available character from stdin and returns it as an
integer\strut
\end{minipage} & \begin{minipage}[t]{0.30\columnwidth}\raggedright
None\strut
\end{minipage} & \begin{minipage}[t]{0.25\columnwidth}\raggedright
Read character\strut
\end{minipage}\tabularnewline
\begin{minipage}[t]{0.11\columnwidth}\raggedright
gets\strut
\end{minipage} & \begin{minipage}[t]{0.23\columnwidth}\raggedright
Reads the next line from stdin and places this into the provided Read
line parameter\strut
\end{minipage} & \begin{minipage}[t]{0.30\columnwidth}\raggedright
Read line\strut
\end{minipage} & \begin{minipage}[t]{0.25\columnwidth}\raggedright
Status\strut
\end{minipage}\tabularnewline
\begin{minipage}[t]{0.11\columnwidth}\raggedright
scanf\strut
\end{minipage} & \begin{minipage}[t]{0.23\columnwidth}\raggedright
Reads the input from the standard input stream stdin and scans that
input according to the format provided\strut
\end{minipage} & \begin{minipage}[t]{0.30\columnwidth}\raggedright
Format, List of Format variables , Value to put the read value
into\strut
\end{minipage} & \begin{minipage}[t]{0.25\columnwidth}\raggedright
Status\strut
\end{minipage}\tabularnewline
\bottomrule
\end{longtable}

When combining the above commands with a loop the behavior of the first
device-driver can be implemented.

\hypertarget{how-can-the-functionality-of-device-driver1-be-implemented}{%
\subsubsection{How can the functionality of device-driver1 be
implemented}\label{how-can-the-functionality-of-device-driver1-be-implemented}}

As specified in the Requirements the functionality of device-driver1
will be to write to stdout.\\
As the application that will be build in this paper will be build with
C, for writing to stdout the basic commands are:

\begin{longtable}[]{@{}llll@{}}
\toprule
\begin{minipage}[b]{0.11\columnwidth}\raggedright
Name\strut
\end{minipage} & \begin{minipage}[b]{0.23\columnwidth}\raggedright
Explanation\strut
\end{minipage} & \begin{minipage}[b]{0.30\columnwidth}\raggedright
Input arguments\strut
\end{minipage} & \begin{minipage}[b]{0.25\columnwidth}\raggedright
Return value\strut
\end{minipage}\tabularnewline
\midrule
\endhead
\begin{minipage}[t]{0.11\columnwidth}\raggedright
putchar\strut
\end{minipage} & \begin{minipage}[t]{0.23\columnwidth}\raggedright
puts the passed character on the screen and returns the same
character\strut
\end{minipage} & \begin{minipage}[t]{0.30\columnwidth}\raggedright
Character to write\strut
\end{minipage} & \begin{minipage}[t]{0.25\columnwidth}\raggedright
Written character\strut
\end{minipage}\tabularnewline
\begin{minipage}[t]{0.11\columnwidth}\raggedright
puts\strut
\end{minipage} & \begin{minipage}[t]{0.23\columnwidth}\raggedright
Writes the provided string and a trailing newline to stdout\strut
\end{minipage} & \begin{minipage}[t]{0.30\columnwidth}\raggedright
String to write\strut
\end{minipage} & \begin{minipage}[t]{0.25\columnwidth}\raggedright
Status\strut
\end{minipage}\tabularnewline
\begin{minipage}[t]{0.11\columnwidth}\raggedright
printf\strut
\end{minipage} & \begin{minipage}[t]{0.23\columnwidth}\raggedright
writes the output to the stdout and produces the output according to the
format provided\strut
\end{minipage} & \begin{minipage}[t]{0.30\columnwidth}\raggedright
Format, List of Format variables to write\strut
\end{minipage} & \begin{minipage}[t]{0.25\columnwidth}\raggedright
Status\strut
\end{minipage}\tabularnewline
\bottomrule
\end{longtable}

The above commands can be used to write to the stdout and can be used to
implemented the second device-driver its behavior.
