\hypertarget{research}{%
\chapter{Research}\label{research}}

\hypertarget{how-will-the-communication-between-processes-take-place-under-minix}{%
\section{How will the communication between processes take place under
Minix}\label{how-will-the-communication-between-processes-take-place-under-minix}}

In this sub-chapter will be explained how a communication between
processes takes place under Minix.

\hypertarget{what-are-messages-and-how-can-they-be-used-to-communicate-between-processes-under-minix}{%
\subsection{What are messages and how can they be used to communicate
between processes under
Minix}\label{what-are-messages-and-how-can-they-be-used-to-communicate-between-processes-under-minix}}

In this sub-question will be explained what messages are and how they
can be send and received under Minix.

\hypertarget{what-are-messages-in-minix}{%
\subsubsection{What are messages in
Minix}\label{what-are-messages-in-minix}}

Minix uses their own message type as payload for sending and receiving
messages. A message are a fixed length of 64 bytes of data, they are
composed of:

\begin{longtable}[]{@{}ll@{}}
\toprule
Name & Explanation\tabularnewline
\midrule
\endhead
Endpoint sender & 4-byte identifier of who sent the
message\tabularnewline
Message type & 4-byte message type identifier\tabularnewline
Payload & 56 bytes of data\tabularnewline
\bottomrule
\end{longtable}

To give further explanation about the message type, each message type
has a different structure of its data.

\hypertarget{what-is-an-endpoint-and-how-can-it-be-determined-under-minix}{%
\subsubsection{What is an endpoint and how can it be determined under
Minix}\label{what-is-an-endpoint-and-how-can-it-be-determined-under-minix}}

The following quote from the minix3 wikipedia explains what an endpoint
is and why it is used in Minix:

\begin{quote}
A endpoint identifies a process uniquely among the operating system. It
is composed of the process slot number concatenated with a generation
number.\\
The reason behind this generation number is that a process slot may be
recycled when a process dies, so unrelated processes may sequentially
share the same process slot number, which would cause problems when
delivering messages to a process which happened to share the same
process slot than a unrelated predecessor.
\end{quote}

The endpoint of a process can be determined in multiple ways, they are:

\begin{longtable}[]{@{}llll@{}}
\toprule
\begin{minipage}[b]{0.11\columnwidth}\raggedright
Name\strut
\end{minipage} & \begin{minipage}[b]{0.23\columnwidth}\raggedright
Explanation\strut
\end{minipage} & \begin{minipage}[b]{0.30\columnwidth}\raggedright
Input arguments\strut
\end{minipage} & \begin{minipage}[b]{0.25\columnwidth}\raggedright
Return value\strut
\end{minipage}\tabularnewline
\midrule
\endhead
\begin{minipage}[t]{0.11\columnwidth}\raggedright
getpid\strut
\end{minipage} & \begin{minipage}[t]{0.23\columnwidth}\raggedright
Determines the endpoint of the process calling the function\strut
\end{minipage} & \begin{minipage}[t]{0.30\columnwidth}\raggedright
None\strut
\end{minipage} & \begin{minipage}[t]{0.25\columnwidth}\raggedright
Endpoint\strut
\end{minipage}\tabularnewline
\begin{minipage}[t]{0.11\columnwidth}\raggedright
\_pm\_findproc\strut
\end{minipage} & \begin{minipage}[t]{0.23\columnwidth}\raggedright
Determines the endpoint of the requested process\strut
\end{minipage} & \begin{minipage}[t]{0.30\columnwidth}\raggedright
Process name, Endpoint (endpoint of process will be inserted into this
parameter)\strut
\end{minipage} & \begin{minipage}[t]{0.25\columnwidth}\raggedright
Status\strut
\end{minipage}\tabularnewline
\bottomrule
\end{longtable}

Besides process specific endpoints there are three special endpoints:

\begin{longtable}[]{@{}lll@{}}
\toprule
\begin{minipage}[b]{0.20\columnwidth}\raggedright
Name\strut
\end{minipage} & \begin{minipage}[b]{0.44\columnwidth}\raggedright
Explanation\strut
\end{minipage} & \begin{minipage}[b]{0.27\columnwidth}\raggedright
Usage, there can be more usages than mentioned below as these are
examples\strut
\end{minipage}\tabularnewline
\midrule
\endhead
\begin{minipage}[t]{0.20\columnwidth}\raggedright
Any\strut
\end{minipage} & \begin{minipage}[t]{0.44\columnwidth}\raggedright
The sender or receiver of the message can be any process\strut
\end{minipage} & \begin{minipage}[t]{0.27\columnwidth}\raggedright
Can be used for broadcasting a message\strut
\end{minipage}\tabularnewline
\begin{minipage}[t]{0.20\columnwidth}\raggedright
None\strut
\end{minipage} & \begin{minipage}[t]{0.44\columnwidth}\raggedright
The sender or receiver of the message can be no process\strut
\end{minipage} & \begin{minipage}[t]{0.27\columnwidth}\raggedright
This endpoint will not be used and the usage is unknown at the time of
writing\strut
\end{minipage}\tabularnewline
\begin{minipage}[t]{0.20\columnwidth}\raggedright
Self\strut
\end{minipage} & \begin{minipage}[t]{0.44\columnwidth}\raggedright
The sender or receiver of the message is the same process\strut
\end{minipage} & \begin{minipage}[t]{0.27\columnwidth}\raggedright
Can be used to simulate messages that would normally be send from other
processes\strut
\end{minipage}\tabularnewline
\bottomrule
\end{longtable}

\hypertarget{how-can-messages-be-send-and-received-under-minix}{%
\subsubsection{How can messages be send and received under
Minix}\label{how-can-messages-be-send-and-received-under-minix}}

First will be discussed what types of message sending and receiving
Minix supports and then how these can be used.

In Minix there are three basic primitives used to send messages, these
are:

\begin{longtable}[]{@{}llll@{}}
\toprule
\begin{minipage}[b]{0.11\columnwidth}\raggedright
Name\strut
\end{minipage} & \begin{minipage}[b]{0.23\columnwidth}\raggedright
Explanation\strut
\end{minipage} & \begin{minipage}[b]{0.30\columnwidth}\raggedright
Input arguments\strut
\end{minipage} & \begin{minipage}[b]{0.25\columnwidth}\raggedright
Return value\strut
\end{minipage}\tabularnewline
\midrule
\endhead
\begin{minipage}[t]{0.11\columnwidth}\raggedright
Send\strut
\end{minipage} & \begin{minipage}[t]{0.23\columnwidth}\raggedright
A message is sent, the sender is blocked until the message is
delivered\strut
\end{minipage} & \begin{minipage}[t]{0.30\columnwidth}\raggedright
Receiver endpoint, Message\strut
\end{minipage} & \begin{minipage}[t]{0.25\columnwidth}\raggedright
Status\strut
\end{minipage}\tabularnewline
\begin{minipage}[t]{0.11\columnwidth}\raggedright
Receive\strut
\end{minipage} & \begin{minipage}[t]{0.23\columnwidth}\raggedright
The process is blocked until a message is delivered to them\strut
\end{minipage} & \begin{minipage}[t]{0.30\columnwidth}\raggedright
Sender endpoint, Message, Status\strut
\end{minipage} & \begin{minipage}[t]{0.25\columnwidth}\raggedright
Status\strut
\end{minipage}\tabularnewline
\begin{minipage}[t]{0.11\columnwidth}\raggedright
Sendrec\strut
\end{minipage} & \begin{minipage}[t]{0.23\columnwidth}\raggedright
A message is sent, the sender is blocked until it receives a reply from
the receiver\strut
\end{minipage} & \begin{minipage}[t]{0.30\columnwidth}\raggedright
Receiver endpoint, Message\strut
\end{minipage} & \begin{minipage}[t]{0.25\columnwidth}\raggedright
Status\strut
\end{minipage}\tabularnewline
\bottomrule
\end{longtable}

\hypertarget{how-can-memory-be-shared-between-processes-under-minix}{%
\subsection{How can memory be shared between processes under
Minix}\label{how-can-memory-be-shared-between-processes-under-minix}}

In this sub-question will be discusses how data can be communicated
between processes using memory grants under Minix.

\hypertarget{what-is-a-memory-grant-under-minix}{%
\subsubsection{What is a memory grant under
Minix}\label{what-is-a-memory-grant-under-minix}}

Memory grants allow processes to share large amount of data, data that
would be inefficient to send with messages as messages only allow a 56
byte payload.

Every type of memory grant share the same three attributes:

\begin{itemize}
\tightlist
\item
  Permitted access mode may be set to read, write or read-write
\item
  Grants are created by an endpoint for an endpoint
\item
  The memory region is a continuous range with byte granularity
\end{itemize}

The granter is the process that created and owns the grant whereas the
grantee is the process to which the memory grant was granted.

Minix supports three types of memory grants:

\begin{longtable}[]{@{}ll@{}}
\toprule
\begin{minipage}[b]{0.30\columnwidth}\raggedright
Name\strut
\end{minipage} & \begin{minipage}[b]{0.64\columnwidth}\raggedright
Explanation\strut
\end{minipage}\tabularnewline
\midrule
\endhead
\begin{minipage}[t]{0.30\columnwidth}\raggedright
Direct grants\strut
\end{minipage} & \begin{minipage}[t]{0.64\columnwidth}\raggedright
A memory region of the granter is accessible by the grantee. The data is
not transferred from the granter to the grantee, the data is only
accessible to the grantee and the granter is still the owner of the
data. This grant is direct because there is no middleman\strut
\end{minipage}\tabularnewline
\begin{minipage}[t]{0.30\columnwidth}\raggedright
Indirect grants\strut
\end{minipage} & \begin{minipage}[t]{0.64\columnwidth}\raggedright
Indirect grants are grants that have been transferred by a grantee : the
grantee delegates its grant to another process. The grantee becomes a
granter and then passes a newly created memory grant credential to the
new grantee. The memory grant thus forms a linked list of indirect
grants, with a direct grant at the start. Indirect grant can themselves
be transferred to another grantee by creating a new indirect
grant.\strut
\end{minipage}\tabularnewline
\begin{minipage}[t]{0.30\columnwidth}\raggedright
Magic grants\strut
\end{minipage} & \begin{minipage}[t]{0.64\columnwidth}\raggedright
Magic grants are grants that are made by one process for memory in
another process. The only service that may create magic grants, is VFS.
VFS uses magic grants to give other services, direct access to memory in
user application processes.\strut
\end{minipage}\tabularnewline
\bottomrule
\end{longtable}

In this paper only the direct grants will be used.

\hypertarget{how-can-memory-grants-be-used-to-share-data-under-minix}{%
\subsubsection{How can memory grants be used to share data under
Minix}\label{how-can-memory-grants-be-used-to-share-data-under-minix}}

As only direct grants will be used in this paper only this type of
memory granting will be further discussed.

In Minix two main function will be used to share data using memory
grants:

\begin{longtable}[]{@{}llll@{}}
\toprule
\begin{minipage}[b]{0.11\columnwidth}\raggedright
Name\strut
\end{minipage} & \begin{minipage}[b]{0.23\columnwidth}\raggedright
Explanation\strut
\end{minipage} & \begin{minipage}[b]{0.30\columnwidth}\raggedright
Input arguments\strut
\end{minipage} & \begin{minipage}[b]{0.25\columnwidth}\raggedright
Return value\strut
\end{minipage}\tabularnewline
\midrule
\endhead
\begin{minipage}[t]{0.11\columnwidth}\raggedright
cpf\_setgrant\_direct\strut
\end{minipage} & \begin{minipage}[t]{0.23\columnwidth}\raggedright
Grants memory to a specific endpoint and returns the grant id\strut
\end{minipage} & \begin{minipage}[t]{0.30\columnwidth}\raggedright
Endpoint grantee, Memory address of data to grant, Memory size in bytes,
Access type\strut
\end{minipage} & \begin{minipage}[t]{0.25\columnwidth}\raggedright
Grant id\strut
\end{minipage}\tabularnewline
\begin{minipage}[t]{0.11\columnwidth}\raggedright
sys\_safecopyfrom\strut
\end{minipage} & \begin{minipage}[t]{0.23\columnwidth}\raggedright
Copies granted memory from a specific endpoint to a memory address\strut
\end{minipage} & \begin{minipage}[t]{0.30\columnwidth}\raggedright
Endpoint granter, Grant id, Grant offset, Memory address to copy data
into, Memory size in bytes, Segment\strut
\end{minipage} & \begin{minipage}[t]{0.25\columnwidth}\raggedright
Status\strut
\end{minipage}\tabularnewline
\bottomrule
\end{longtable}

The Access type specifies what kind of access the grantee has over the
memory, either read, write or map.

These two functions combined with sending and receiving messages allow
for sharing data between processes under Minix.
