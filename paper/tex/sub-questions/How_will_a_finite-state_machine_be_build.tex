\hypertarget{research}{%
\chapter{Research}\label{research}}

\hypertarget{how-will-a-finite-state-machine-be-build}{%
\section{How will a finite-state machine be
build}\label{how-will-a-finite-state-machine-be-build}}

In this sub-chapter will be explained how a finite-state machine will be
build.

\hypertarget{how-will-a-finite-state-machine-save-its-transition-table-sensitivity-list-and-alphabet}{%
\subsection{How will a finite-state machine save its transition-table,
sensitivity-list and
alphabet}\label{how-will-a-finite-state-machine-save-its-transition-table-sensitivity-list-and-alphabet}}

In this section will be explained how a finite-state machine saves its
information.

\hypertarget{how-will-a-finite-state-machine-save-its-transition-table}{%
\subsubsection{How will a finite-state machine save its
transition-table}\label{how-will-a-finite-state-machine-save-its-transition-table}}

A transition-table stores the information that is needed for a
finite-state machine to make transitions, determine if it can execute an
action and what the next state will be after executing a certain action
from a certain state.

Because the application that will be build in this paper is supposed to
be real-time, and therefore execution time is limited, the
transition-table will have the following requirements:

\begin{itemize}
\tightlist
\item
  Store the states.
\item
  Store the actions.
\item
  Can be used to determine the next state given the current state and
  the to be executed action.
\item
  Relative fast lookup time.
\end{itemize}

This leads to the following design: The transition-table will consist of
a two-dimensional array, where the index of the first array is the
current state, and the index of the second array is the to be executed
action.\\
The result will be in range of -1 to the maximum value of an integer. Is
the result -1 then the to be executed action is not a valid action from
the current state, any other value is the new state the finite-state
machine will be in after executing the action.\\
The size of the outer array, the array using state as index, will be the
sum of the total amount of states in the fsp. The size of the inner
arrays, the arrays using actions as indexes, will be the sum of the
total amount of actions in the fsp.

This design has the following benefits:

\begin{itemize}
\tightlist
\item
  \(O (1)\) lookup time, because indexes in arrays are used, this means
  that the time to determine the next state and if an action can be
  executed is relative short.
\item
  The sensitivity-list can be quickly derived from the transition-table.
  TODO: add ref to sensitivity-list
\end{itemize}

This design has one disadvantage, an amount of data is not used, while
still being allocated, and therefore wasted. In a more advanced language
a hash table would have been used, and this disadvantage would not
apply.

\hypertarget{how-will-a-finite-state-machine-save-its-sensitivity-list}{%
\subsubsection{How will a finite-state machine save its
sensitivity-list}\label{how-will-a-finite-state-machine-save-its-sensitivity-list}}

The sensitivity-list is used to determine if a finite-state machine can
execute an action from the current state and contains all the actions
that can be executed from the current state the finite-state machine is
in.\\
The sensitivity-list will be derived from the transition-table.

The sensitivity-list has the following requirements:

\begin{itemize}
\tightlist
\item
  Store the actions that can be executed in the current state.
\item
  Relative fast lookup time.
\end{itemize}

This leads to the following design: The sensitivity-list will consist of
an one-dimensional array where the index is the to be executed action
and the result will determine if the finite-state machine can execute
the action.\\
The result will be in range of -1 to the maximum value of an integer. Is
the result -1 then the to be executed action is not a valid action from
the current state, any other value means that the action can be
executed.

The design of the sensitivity-list is similar to that of the inner
arrays of the transition-table, and has therefore the same advantages
and disadvantages as the transition-table.

\hypertarget{how-will-a-finite-state-machine-save-its-alphabet}{%
\subsubsection{How will a finite-state machine save its
alphabet}\label{how-will-a-finite-state-machine-save-its-alphabet}}

The alphabet is used to determine if a finite-state machine can ever
execute an action and should contain all of the actions that a
finite-state machine could ever execute.\\
The alphabet will be derived from the transition-table.

The alphabet has the following requirements:

\begin{itemize}
\tightlist
\item
  Store the actions that the finite-state machine is ever capable of
  executing.
\item
  Relative fast lookup time.
\end{itemize}

This leads to the following design: The alphabet will consist of an
one-dimensional array where the index is the to be executed action and
the result will determine if the finite-state machine can execute the
action.\\
The result will be in range of -1 to the maximum value of an integer. Is
the result -1 then the to be executed action is not a valid action
cannot ever be executed, any other value means that the action can be
executed.

The design of the alphabet is similar to that of the inner arrays of the
transition-table, and has therefore the same advantages and
disadvantages as the transition-table.

\hypertarget{how-will-a-finite-state-machine-determine-its-sensitivity-list-and-alphabet}{%
\subsection{How will a finite-state machine determine its
sensitivity-list and
alphabet}\label{how-will-a-finite-state-machine-determine-its-sensitivity-list-and-alphabet}}

As described above the design of the sensitivity-list and alphabet are
similar to the inner arrays of the transaction-table.\\
All of them return either an -1, not a viable action, or a value in the
range from 0 to the maximum value of an integer, is a viable action,
when using an action as index.

\hypertarget{how-will-a-finite-state-machine-determine-its-sensitivity-list}{%
\subsubsection{How will a finite-state machine determine its
sensitivity-list}\label{how-will-a-finite-state-machine-determine-its-sensitivity-list}}

The sensitivity-list is equal to the inner array that is returned when
inserting the current state in the outer array of the
transition-table.\\
This is because when using the current state as the index on the outer
array of the transition-table, an array is returned that contains all
the actions that can be executed from the current state, and this is the
sensitivity-list.

The following steps will be taken to determine the sensitivity-list:

\begin{itemize}
\tightlist
\item
  Copy the inner array that is returned when inserting the current state
  in the outer array of the transition-table into the sensitivity-list.
\end{itemize}

\hypertarget{how-will-a-finite-state-machine-determine-its-alphabet}{%
\subsubsection{How will a finite-state machine determine its
alphabet}\label{how-will-a-finite-state-machine-determine-its-alphabet}}

The alphabet contains all the actions that a finite-state machine can
execute from any state. Each action in the transition-table must be
checked if it can ever be executed, and if so, add it to the alphabet.

The following steps will be taken to determine the alphabet:

\begin{itemize}
\tightlist
\item
  Create an alphabet with all values on False.
\item
  Iterate over every action in the transition-table.
\item
  If an action can be executed set that action in the alphabet on True.
\end{itemize}

\hypertarget{how-will-a-finite-state-machine-make-transitions}{%
\subsubsection{How will a finite-state machine make
transitions}\label{how-will-a-finite-state-machine-make-transitions}}

Making a transition is to execute an action, determine a new
sensitivity-list and new current state, and setting the new
sensitivity-list and current state.

How a action is executed and the new sensitivity-list is determined is
dependent on what kind of finite-state machine is making the transition.
There are two different kind of transitions, a Thread transition, and a
device-driver transition.

\textbf{Thread transition:}\\
The new sensitivity-list is a direct consequence of the fsp design.

\textbf{Device-driver transition:}\\
The new sensitivity-list is either direct consequence of the fsp design
or a consequence of the hardware or sensor data the device-driver
operates upon.

This leads to the following steps of making a transition:

\begin{itemize}
\tightlist
\item
  Determine if the action can be executed, using the transition-table,
  current state and to be executed action.
\item
  Executing the action, this will return the new sensitivity-list, this
  could be either the Thread implementation or the device-driver
  implementation. An empty sensitivity-list could also be returned
  indicating that the action could not successfully execute.
\item
  Checking if the new sensitivity-list is not empty, if it is empty not
  proceed.
\item
  Determine the new state of the finite-state machine using its
  transition-table, current state and executed action.
\item
  Set the finite-state machine its new state and the new
  sensitivity-list.
\end{itemize}
