\hypertarget{how-will-the-functionality-of-the-device-drivers-be-implemented}{%
\section{How will the functionality of the device-drivers be
implemented}\label{how-will-the-functionality-of-the-device-drivers-be-implemented}}

\hypertarget{how-will-the-functionality-of-device-driver0-be-implemented}{%
\subsection{How will the functionality of device-driver0 be
implemented}\label{how-will-the-functionality-of-device-driver0-be-implemented}}

As specified in the Requirements the functionality of device-driver0
will be to continuously read from stdin.\\
As the application that will be build in this paper will be build with
C, for reading the stdin the basic commands are:

\begin{itemize}
\tightlist
\item
  Name ; Explanation ; Input arguments ; Return value
\item
  getchar ; Reads the next available character from stdin and returns it
  as an integer ; None ; Read character
\item
  gets ; Reads the next line from stdin and places this into the
  provided Read line parameter ; Read line ; Status
\item
  scanf ; Reads the input from the standard input stream stdin and scans
  that input according to the format provided ; Format, List of Format
  variables , Value to put the read value into ; Status
\end{itemize}

Using the above commands the behavior of the first device-driver can be
implemented.

\hypertarget{how-will-the-functionality-of-device-driver1-be-implemented}{%
\subsection{How will the functionality of device-driver1 be
implemented}\label{how-will-the-functionality-of-device-driver1-be-implemented}}

As specified in the Requirements the functionality of device-driver1
will be to write to stdout.\\
As the application that will be build in this paper will be build with
C, for writing to stdout the basic commands are:

\begin{itemize}
\tightlist
\item
  Name ; Explanation ; Input arguments ; Return value
\item
  putchar ; puts the passed character on the screen and returns the same
  character ; Character to write ; Written character
\item
  puts ; Writes the provided string and a trailing newline to stdout ;
  String to write ; Status
\item
  printf ; writes the output to the stdout and produces the output
  according to the format provided ; Format, List of Format variables to
  write ; Status
\end{itemize}

Using the above commands the behavior of the second device-driver can be
implemented.
