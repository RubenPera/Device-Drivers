\hypertarget{implementation}{%
\chapter{Implementation}\label{implementation}}

\hypertarget{how-will-a-synchronization-server-be-build}{%
\section{How will a synchronization-server be
build}\label{how-will-a-synchronization-server-be-build}}

In this section the implementation of the synchronization-server will be
given.

\hypertarget{how-will-the-synchronization-server-save-the-data-of-the-finite-state-machines}{%
\subsection{How will the synchronization-server save the data of the
finite-state
machines}\label{how-will-the-synchronization-server-save-the-data-of-the-finite-state-machines}}

For storing the finite-state machine and finite-state machine derivative
and their attributes structs will be used:

\begin{Shaded}
\begin{Highlighting}[]
\KeywordTok{typedef} \KeywordTok{struct}
\NormalTok{\{}
    \DataTypeTok{int}\NormalTok{ sensitivity_list[ACTIONS];}
    \DataTypeTok{int}\NormalTok{ alphabet[ACTIONS];}
    \DataTypeTok{int}\NormalTok{ endpoint;}
    \DataTypeTok{int}\NormalTok{ index;}
\NormalTok{\} finite_state_machine_node_derivative_t;}

\KeywordTok{typedef} \KeywordTok{struct}
\NormalTok{\{}
    \DataTypeTok{int}\NormalTok{ endpoint;}
\NormalTok{    finite_state_machine_node_derivative_t *derivatives[DERIVATIVES];}
\NormalTok{\} synchronization_server_t;}
\end{Highlighting}
\end{Shaded}

\hypertarget{how-will-the-synchronization-server-determine-the-next-action-for-the-finite-state-machines}{%
\subsection{How will the synchronization-server determine the next
action for the finite-state
machines}\label{how-will-the-synchronization-server-determine-the-next-action-for-the-finite-state-machines}}

For determining the next action that the finite-state machines will
execute, the following code snippet is used:

\begin{Shaded}
\begin{Highlighting}[]
\DataTypeTok{int}\NormalTok{ next_action(}
\NormalTok{    synchronization_server_t *self)}
\NormalTok{\{}

    \DataTypeTok{int}\NormalTok{ actions[ACTIONS] = \{NO_ACTION\};}

\NormalTok{    get_combined_actions(self, actions);}

\NormalTok{    get_valid_actions(self, actions);}

    \ControlFlowTok{return}\NormalTok{ random_action_from_actions(actions);}
\NormalTok{\}}

\DataTypeTok{void}\NormalTok{ get_valid_actions(}
\NormalTok{    synchronization_server_t *self,}
    \DataTypeTok{int}\NormalTok{ actions[ACTIONS])}
\NormalTok{\{}
    \DataTypeTok{int}\NormalTok{ a = }\DecValTok{0}\NormalTok{, d = }\DecValTok{0}\NormalTok{;}

    \DataTypeTok{int}\NormalTok{ actions_to_remove[ACTIONS] = \{HAS_ACTION\};}

    \ControlFlowTok{for}\NormalTok{ (a = }\DecValTok{0}\NormalTok{; a < ACTIONS; a++)}
\NormalTok{    \{}
        \ControlFlowTok{for}\NormalTok{ (d = }\DecValTok{0}\NormalTok{; d < DERIVATIVES; d++)}
\NormalTok{        \{}
            \ControlFlowTok{if}\NormalTok{ (self->derivatives[d]->alphabet[a] != NO_ACTION &&}
\NormalTok{                self->derivatives[d]->sensitivity_list[a] == NO_ACTION)}
\NormalTok{            \{}
\NormalTok{                actions_to_remove[a] = NO_ACTION;}
\NormalTok{            \}}
\NormalTok{        \}}
\NormalTok{    \}}

    \ControlFlowTok{for}\NormalTok{ (a = }\DecValTok{0}\NormalTok{; a < ACTIONS; a++)}
\NormalTok{    \{}
        \ControlFlowTok{if}\NormalTok{ (actions_to_remove[a] == NO_ACTION)}
\NormalTok{        \{}
\NormalTok{            actions[a] = NO_ACTION;}
\NormalTok{        \}}
\NormalTok{    \}}
\NormalTok{\}}

\DataTypeTok{void}\NormalTok{ get_combined_actions(}
\NormalTok{    synchronization_server_t *self,}
    \DataTypeTok{int}\NormalTok{ combined_actions[ACTIONS])}
\NormalTok{\{}
    \DataTypeTok{int}\NormalTok{ d = }\DecValTok{0}\NormalTok{, a = }\DecValTok{0}\NormalTok{;}

    \ControlFlowTok{for}\NormalTok{ (d = }\DecValTok{0}\NormalTok{; d < DERIVATIVES; d++)}
\NormalTok{    \{}
        \ControlFlowTok{for}\NormalTok{ (a = }\DecValTok{0}\NormalTok{; a < ACTIONS; a++)}
\NormalTok{        \{}
            \ControlFlowTok{if}\NormalTok{ (self->derivatives[d]->sensitivity_list[a] != NO_ACTION &&}
\NormalTok{                combined_actions[a] == NO_ACTION)}
\NormalTok{            \{}
\NormalTok{                combined_actions[a] = HAS_ACTION;}
\NormalTok{            \}}
\NormalTok{        \}}
\NormalTok{    \}}
\NormalTok{\}}

\DataTypeTok{int}\NormalTok{ random_action_from_actions(}
    \DataTypeTok{int}\NormalTok{ actions[ACTIONS])}
\NormalTok{\{}
    \DataTypeTok{int}\NormalTok{ i = }\DecValTok{0}\NormalTok{, selected_action = NO_ACTION;}

    \ControlFlowTok{for}\NormalTok{ (i = rand(); i < ACTIONS; i++)}
\NormalTok{    \{}
        \ControlFlowTok{if}\NormalTok{ (actions[i] != NO_ACTION)}
\NormalTok{        \{}
\NormalTok{            selected_action = i;}
            \ControlFlowTok{break}\NormalTok{;}
\NormalTok{        \}}
\NormalTok{    \}}

    \ControlFlowTok{return}\NormalTok{ selected_action == NO_ACTION ? random_action_from_actions(actions) : selected_action;}
\NormalTok{\}}
\end{Highlighting}
\end{Shaded}

In \emph{get\_combined\_actions} an array of all the actions is
determined. In \emph{get\_valid\_actions} the actions that are able to
be executed by the current derivates is determined. In
\emph{random\_action\_from\_actions} a random action is chosen from the
array of actions, this action will be executed.

\hypertarget{how-will-the-synchronization-server-determine-the-finite-state-machines-that-will-execute-an-action}{%
\subsection{How will the synchronization-server determine the
finite-state machines that will execute an
action}\label{how-will-the-synchronization-server-determine-the-finite-state-machines-that-will-execute-an-action}}

For determining the derivatives that will execute an action the
following code snippet is used:

\begin{Shaded}
\begin{Highlighting}[]
\DataTypeTok{void}\NormalTok{ get_executable_derivatives(}
\NormalTok{    synchronization_server_t *self,}
    \DataTypeTok{int}\NormalTok{ derivatives[DERIVATIVES],}
    \DataTypeTok{int}\NormalTok{ action)}
\NormalTok{\{}
    \DataTypeTok{int}\NormalTok{ d = }\DecValTok{0}\NormalTok{;}

    \ControlFlowTok{for}\NormalTok{ (d = }\DecValTok{0}\NormalTok{; d < DERIVATIVES; d++)}
\NormalTok{    \{}
        \ControlFlowTok{if}\NormalTok{ (self->derivatives[d]->alphabet[action] != NO_ACTION &&}
\NormalTok{            self->derivatives[d]->sensitivity_list[action] != NO_ACTION)}
\NormalTok{        \{}
\NormalTok{            derivatives[action] = HAS_ACTION;}
\NormalTok{        \}}
\NormalTok{    \}}
\NormalTok{\}}
\end{Highlighting}
\end{Shaded}

The array of derivatives is iterated and only the derivatives that have
the action in their alphabet and sensitivity\_list will execute the
action.
