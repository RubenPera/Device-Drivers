\hypertarget{implementation}{%
\section{Implementation}\label{implementation}}

\hypertarget{how-will-the-communication-between-a-finite-state-machine-and-the-synchronization-server-take-place}{%
\subsection{How will the communication between a finite-state machine
and the synchronization-server take
place}\label{how-will-the-communication-between-a-finite-state-machine-and-the-synchronization-server-take-place}}

In this section the implementation of the communication between
finite-state machine and synchronization-server will be given.

\hypertarget{how-will-a-finite-state-machine-send-their-status-to-the-synchronization-server}{%
\subsubsection{How will a finite-state machine send their status to the
synchronization-server}\label{how-will-a-finite-state-machine-send-their-status-to-the-synchronization-server}}

In this section the implementation of the finite-state machine sending
its status to the synchronization-server be given.

\hypertarget{what-is-a-finite-state-machine-status-and-how-will-it-be-stored}{%
\paragraph{What is a finite-state machine status and how will it be
stored}\label{what-is-a-finite-state-machine-status-and-how-will-it-be-stored}}

For storing the finite-state machine status and its attributes a struct
will be used:

\begin{Shaded}
\begin{Highlighting}[]
\KeywordTok{typedef} \KeywordTok{struct}
\NormalTok{\{}
    \DataTypeTok{int}\NormalTok{ endpoint;}
    \DataTypeTok{int}\NormalTok{ index;}
    \DataTypeTok{int}\NormalTok{ sensitivity_list[ACTIONS];}
    \DataTypeTok{int}\NormalTok{ alphabet[ACTIONS];}

\NormalTok{\} finite_state_machine_node_status_t;}
\end{Highlighting}
\end{Shaded}

\hypertarget{how-will-a-finite-state-machine-status-be-communicated-with-the-synchronization-server}{%
\paragraph{How will a finite-state machine status be communicated with
the
synchronization-server}\label{how-will-a-finite-state-machine-status-be-communicated-with-the-synchronization-server}}

The finite-state machine node will use the following code snippet for
sending its finite-state machine node status to the
synchronization-server.

\begin{Shaded}
\begin{Highlighting}[]
\DataTypeTok{void}\NormalTok{ node_send_status(finite_state_machine_node_t *self)}
\NormalTok{\{}
\NormalTok{    finite_state_machine_node_status_t *status;}

\NormalTok{    status = finite_state_machine_node_status_init(self);}

\NormalTok{    send_grant_and_block(}
\NormalTok{        self->endpoint,}
\NormalTok{        self->send_status_endpoint,}
\NormalTok{        status,}
        \KeywordTok{sizeof}\NormalTok{(status))}
\NormalTok{\}}
\end{Highlighting}
\end{Shaded}

The send\_grant\_and\_block is an abstraction method written for
creating and sending grants and is included in the appendix.

The synchronization-server will use the following code snippet for
receiving a finite-state machine node status.

\begin{Shaded}
\begin{Highlighting}[]
\DataTypeTok{void}\NormalTok{ synchronization_server_receive_status(}
\NormalTok{    synchronization_server_t *self)}
\NormalTok{\{}
\NormalTok{    finite_state_machine_node_derivative_t *derivative;}

\NormalTok{    finite_state_machine_node_status_t *status }\DataTypeTok{int}\NormalTok{ index = NO_ACTION;}

\NormalTok{    status = receive_status();}

\NormalTok{    derivative = synchronization_server_status_to_derivative(status);}

    \ControlFlowTok{if}\NormalTok{ (derivative->index == NO_INDEX)}
\NormalTok{    \{}
\NormalTok{        index = synchronization_server_add_derivative(self, derivative);}
\NormalTok{        derivative_send_index(derivative, self->endpoint, index);}
\NormalTok{    \}}
    \ControlFlowTok{else}
\NormalTok{    \{}
\NormalTok{        synchronization_server_set_derivative(self, derivative);}
\NormalTok{    \}}
\NormalTok{\}}
\end{Highlighting}
\end{Shaded}

\hypertarget{how-will-the-synchronization-server-communicate-actions-to-the-finite-state-machines}{%
\subsubsection{How will the synchronization-server communicate actions
to the finite-state
machines}\label{how-will-the-synchronization-server-communicate-actions-to-the-finite-state-machines}}

The synchronization-server uses the following code snippet for sending
the determined action to the derivatives.

\begin{Shaded}
\begin{Highlighting}[]
\DataTypeTok{void}\NormalTok{ send_action_to_derivatives(synchronization_server_t *self,}
    \DataTypeTok{int}\NormalTok{ derivatives[DERIVATIVES),}
    \DataTypeTok{int}\NormalTok{ action )}
\NormalTok{\{}
    \DataTypeTok{int}\NormalTok{ i = }\DecValTok{0}\NormalTok{;}

    \ControlFlowTok{for}\NormalTok{ (i = }\DecValTok{0}\NormalTok{; i < ACTIONS; i++)}
\NormalTok{    \{}
        \ControlFlowTok{if}\NormalTok{ (derivatives[i] != NO_ACTION)}
\NormalTok{        \{}
\NormalTok{            derivative_send_action(self->derivatives[i], self->endpoint, action);}
\NormalTok{        \}}
\NormalTok{    \}}
\NormalTok{\}}

\DataTypeTok{void}\NormalTok{ derivative_send_action(}
\NormalTok{    finite_state_machine_node_derivative_t *self,}
    \DataTypeTok{int}\NormalTok{ sender,}
    \DataTypeTok{int}\NormalTok{ action)}
\NormalTok{\{}
\NormalTok{    send_action(sender, self->endpoint, action);}
\NormalTok{\}}
\end{Highlighting}
\end{Shaded}

The finite-state machine node uses the following code snippet for
receiving the action.

\begin{Shaded}
\begin{Highlighting}[]
\DataTypeTok{void}\NormalTok{ driver_receive_action(}
\NormalTok{    finite_state_machine_node_t *self)}
\NormalTok{\{}
    \DataTypeTok{int}\NormalTok{ action;}

\NormalTok{    action = receive_action();}
\NormalTok{    make_transition(self->fsm, action);}
\NormalTok{    node_send_status(self);}
\NormalTok{\}}
\end{Highlighting}
\end{Shaded}

