\hypertarget{how-will-a-finite-state-machine-be-build}{%
\section{How will a finite-state machine be
build}\label{how-will-a-finite-state-machine-be-build}}

In this section the implementation of the finite-state machine will be
given.

\hypertarget{how-will-a-finite-state-machine-save-its-transition-table-sensitivity-list-and-alphabet}{%
\subsection{How will a finite-state machine save its transition-table,
sensitivity-list and
alphabet}\label{how-will-a-finite-state-machine-save-its-transition-table-sensitivity-list-and-alphabet}}

For storing the finite-state machine and its attributes a struct will be
used:

\begin{verbatim}
typedef struct
{
    int transition_table[STATES][ACTIONS];
    int sensitivity_list[ACTIONS];
    int alphabet[ACTIONS];
    int cur_state;

} finite_state_machine_t;
\end{verbatim}

STATES are the total amount of states and ACTIONS are the total amount
of actions. As C does not support arrays of undetermined sizes, the
total amount of states and actions must be known during compilation,
these and other constant values are stored in one shared constant file.

\hypertarget{how-will-a-finite-state-machine-determine-its-sensitivity-list-and-alphabet}{%
\subsection{How will a finite-state machine determine its
sensitivity-list and
alphabet}\label{how-will-a-finite-state-machine-determine-its-sensitivity-list-and-alphabet}}

In this section the implementation of determining the sensitivity-list
and alphabet will be provided.

\hypertarget{how-will-a-finite-state-machine-determine-its-sensitivity-list}{%
\subsubsection{How will a finite-state machine determine its
sensitivity-list}\label{how-will-a-finite-state-machine-determine-its-sensitivity-list}}

For determining the finite-state machine its sensitivity-list the
following code snippet is used:

\begin{verbatim}
void fsm_set_sensitivity_list(
    finite_state_machine_t *self,
    int sensitivity_list[ACTIONS])
{
    array_copy(
        (int *)self->sensitivity_list, 
        (int *)self->transition_table[self->cur_state]);
}
\end{verbatim}

The used function array\_copy copies the values of one array to the
other array and will be included in the appendix.

\hypertarget{how-will-a-finite-state-machine-determine-its-alphabet}{%
\subsubsection{How will a finite-state machine determine its
alphabet}\label{how-will-a-finite-state-machine-determine-its-alphabet}}

For determining the finite-state machine its alphabet the following code
snippet is used:

\begin{verbatim}
void fsm_set_alphabet(
    finite_state_machine_t *self)
{
    int s = 0, a = 0;

    for (s = 0; s < STATES; s++)
    {
        for (a = 0; a < 0; a++)
        {
            if (self->transition_table[s][a] != NO_ACTION &&
                self->alphabet[a] == NO_ACTION)
            {
                self->alphabet[a] = HAS_ACTION;
            }
        }
    }
}
\end{verbatim}

\hypertarget{how-will-a-finite-state-machine-make-transitions}{%
\subsection{How will a finite-state machine make
transitions}\label{how-will-a-finite-state-machine-make-transitions}}

For making transitions the finite-state machine uses the following code
snippet.

\begin{verbatim}
void make_transition(
    finite_state_machine_t *self,
    int action)
{
    int sensitivity_list[ACTIONS];

    if (fsm_can_execute_action(self, action))
    {
        execute_action(self, action, sensitivity_list);

        if (sensitivity_list != NULL)
        {
            array_copy(self->sensitivity_list, sensitivity_list);
            self->cur_state = fsm_next_state(self, action);
        }
    }
}

int fsm_can_execute_action(
    finite_state_machine_t *self,
    int action)
{
    return fsm_next_state(self, action) != NO_ACTION;
}

int fsm_next_state(
    finite_state_machine_t *self,
    int action)
{
    return self->transition_table[self->cur_state][action];
}
\end{verbatim}

However, the function make\_transition has to implemented by each
finite-state machine that uses it, as the function execute\_action is
different for every finite-state machine implementation.
