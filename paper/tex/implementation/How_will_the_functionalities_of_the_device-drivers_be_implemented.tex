\hypertarget{how-will-the-functionality-of-the-device-drivers-be-implemented}{%
\section{How will the functionality of the device-drivers be
implemented}\label{how-will-the-functionality-of-the-device-drivers-be-implemented}}

In this section the implemented of the two device-drivers will be given.
In the FSP model the name stdin\_driver was used for device-driver0 and
stdout\_driver for device-driver1, these names will also be used in this
paper.

When executing an action the device-drivers execute the specified action
and determine their new sensitivity-list.

\hypertarget{how-is-the-functionality-of-stdin_driver-implemented}{%
\subsection{How is the functionality of stdin\_driver
implemented}\label{how-is-the-functionality-of-stdin_driver-implemented}}

As specified in the Requirements the functionality of stdin\_driver will
be to read from stdin. The following code snippet is used:

\begin{verbatim}
void execute_action(
    finite_state_machine_t *self,
    int action,
    int sensitivity_list[ACTIONS])
{

    fsm_set_sensitivity_list_empty(sensitivity_list);

    switch (action)
    {
    case CHECK_STDIN_VALUE:
        action_check_stdin_value(sensitivity_list);
        break;

    case W:
        array_copy(
            sensitivity_list, 
            action_key(self->sensitivity_list, action));
        break;

    case A:
        array_copy(
            sensitivity_list, 
            action_key(self->sensitivity_list, action));
        break;

    case S:
        array_copy(
            sensitivity_list, 
            action_key(self->sensitivity_list, action));
        break;

    case D:
        array_copy(
            sensitivity_list, 
            action_key(self->sensitivity_list, action));
        break;

    default:
        break;
    }
}

int * action_key(int sensitivity_list[ACTIONS], int action)
{
    return sensitivity_list[action];
}

void action_check_stdin_value(int sensitivity_list[ACTIONS])
{
    int key = KEY_S;
    int input = 0;
    int i = 0;

    input = getchar();

    for (i = 0; i < KEYS; i++)
    {
        if (input == ALLOWED_KEYS[i])
        {
            key = KEY_ACTIONS[i];
        }
    }

    sensitivity_list[key] = HAS_ACTION;
}
\end{verbatim}

\hypertarget{how-can-the-functionality-of-stdout_driver-be-implemented}{%
\subsection{How can the functionality of stdout\_driver be
implemented}\label{how-can-the-functionality-of-stdout_driver-be-implemented}}

As specified in the Requirements the functionality of stdout\_driver
will be to write to stdout. The following code snippet is used:

\begin{verbatim}
void execute_action(
        finite_state_machine_t *self,
        int action,
        int sensitivity_list[ACTIONS]) {

    switch (action){
        case FORWARD:
            array_copy(
                sensitivity_list, 
                action_direction(self->sensitivity_list, FORWARD, 'F');
            break;

        case LEFT:
            array_copy(
                sensitivity_list, 
                action_direction(self->sensitivity_list, LEFT, 'L');
            break;

        case BACK:
            array_copy(
                sensitivity_list, 
                action_direction(self->sensitivity_list, BACK, 'B');
            break;

        case RIGHT:
            array_copy(
                sensitivity_list, 
                action_direction(self->sensitivity_list, RIGHT, 'R');
            break;

        case WRITE_STDOUT:
            array_copy(
                sensitivity_list, 
                action_write_stdout(self->sensitivity_list, WRITE_STDOUT);
        break;

        default:
            break;
    }
}

int * action_direction(int sensitivity_list[ACTIONS], int action, char direction)
{
    value_to_write = direction;
    return sensitivity_list[action];
}

void action_write_stdout(int sensitivity_list[ACTIONS], int action)
{
    printf("%c \n", value_to_write);
    return sensitivity_list[action];
}
\end{verbatim}
